% This is a simple sample document.  For more complicated documents take a look in the exercise tab. Note that everything that comes after a % symbol is treated as comment and ignored when the code is compiled.

\documentclass{article} % \documentclass{} is the first command in any LaTeX code.  It is used to define what kind of document you are creating such as an article or a book, and begins the document preamble
\usepackage{amssymb}
\usepackage{amsmath} % \usepackage is a command that allows you to add functionality to your LaTeX code

\title{Rapport de Stage} % Sets article title
\author{Vincent Callegari} % Sets authors name
\date{\today} % Sets date for date compiled
\usepackage{graphicx}
\graphicspath{{./images/graph/}}
% The preamble ends with the command \begin{document}
	\begin{document}
		\maketitle
		\newpage
		\tableofcontents
		\newpage
		\section{Introduction}
		
		\subsection{la problèmatique}
		Le but de ce stage est de trouver des trajectoires viable de satellite pour observer la couronne solaire en utilisant la Lune comme corps occultant, le tout avec comme système de propulsion une voile solaire.
		
		La voile solaire fonctionne de la même manière qu'une voile normale à l'exception qu'elle utilise la quantité de mouvement de la lumière au lieu de la quantité de mouvement de l'air pour changer de vitesse. Les forces générée par des voiles solaires ne peuvent pas se diriger dans toutes les directions et sont très faible en comparaison des formes de propulsion les plus utilisé par les véhicules spatiaux mais ont le grand avantage de ne pas dépenser de carburant.
		
		L'observation de la couronne solaire est compliqué car elle est bien moins lumineuse que le soleil lui même. Afin d'observer cette couronne, il est donc nécessaire d'occulter le disque solaire, plus l'objet utilisé pour occulter le soleil est grand et éloigné, plus l'observation peut être précise. Les planètes, planètes naines et autre astre sphérique sont donc des objets idéal pour effectuer des mesures car leurs taille excède largement tout objet occultant que l'humanité peut construire à ce jour.
		
		L'objet du système solaire qui semble le plus indiqué pour être utilisé comme astre occultant est la Lune car son absence d'atmosphère significative et sa grande proximité avec la Terre combiné avec de suffisamment petite taille pour pouvoir effectuer des observation à distance relativement faible (environ égale à la distance terre-lune).
		
		Sur Terre il est possible d'effectuer des observation de la couronne solaire durant les éclipses totales où la Lune joue le rôle d'objet occultant. Cependant ces observations ont lieu en moyenne une fois tout les 18 mois et ne permettent des observations de quelques minutes à peine. L'idée serait de placer un satellite sur des trajectoires qui permettent de faire des observations plus souvent et pendant plus longtemps. 
		 
		 Dans ce rapport, je vais détailler le travail que j'ai effectué et les conclusions et résultats auxquels je suis arrivé ainsi que les compétence acquise au cours du stage.
		 
		 \newpage
		\section{travail demandé}
		\subsection{objectif du stage}
		Le travail qui m'a été demandé durant ce stage regroupait plusieurs tâche : tout d'abord, étudier comment se déplaçait la zone d'observation de la Lune  au cours du temps ( la zone d'observation étant où le soleil est caché par la lune mais où la couronne solaire reste visible). Ensuite il fallait , à partir de ces résultats, essayer de trouver des trajectoires qui permettraient de traverser la zone d'observation de manière répété et de rester le plus longtemps possible à l'intérieur de cette zone lorsque l'on la traversait.
		
		Étant donné qu'il y aura sans doute de nombreuses forces perturbatrices et qu'une trajectoire passive (un objet sans contrôle qui ne subit que les forces de gravité de la Terre, Lune, Soleil etc ...) ne traversera sans doute pas la zone d'observation très souvent, il est nécessaire d'ajouter au système un contrôle de trajectoire sous la forme d'une voile solaire. Cette voile permettrait au satellite de changer et de corriger sa trajéctoire afin de maximiser les observations.
		
		Il s'agit d'un problème de contrôle optimal.
		
		Le but final du projet est de présenter un rapport final rapportant tous les résultats et observations faîtes lors du stages ainsi que des programmes capable de calculer différentes valeur utile (zone d'observation, trajectoire de satellite ou contrôle de ce dernier)
		
		\subsection{tâches}
		En résumé mon travail se répartit en plusieurs étapes décrite ci après:
		\begin{itemize}
			\item calculer la trajectoire de la zone d'observation
			\item trouver des trajectoires passive (sans contrôle) qui permettent des observations
			\item trouver des trajectoires active (avec contrôle) qui permettent des observations répétés 
			\item résumer mes résultats dans un rapport
		\end{itemize}
		
		De plus le modèle utilisé pour résoudre ce problème sera d'abord implémenté dans une version simplifiée : simple trajectoire circulaire de la lune avec vitesse fixe en 2D. Puis gagneras en complexité au fur et à mesure que le projet avanceras jusqu'à être très proche de la réalité (prise en compte de la variation de l'orbite lunaire causé par le soleil par exemple).
		
		\newpage
		\section{travail effectué}
		
		\subsection{première observations}
		Avant de commencer à travailler sur le problème, il fallait que j'en apprennent plus sur les notions du projet, j'ai donc lu divers articles sur des projets d'occultation solaire à l'aide de la Terre (utilisée comme astre occultant), une thèses sur les voiles solaires, ainsi qu'un cours sur le contrôle optimal car je n'ai pas encore eu de cours sur ce sujet.
		
		J'ai commencé par analyser le déplacement de la zone d'observation de la Lune, j'ai utilisé comme base des travaux qui avait déjà été effectué en utilisant la Terre comme objet occultant. J'ai utiliser des notions basiques de géométrie pour déterminer où se trouvait la zone d'observation de la Lune en fonction de sa position, j'ai également déterminé quel paramètres pouvaient être négligés afin de rendre le calcul plus simple.
		
		J'ai ensuite recherché une bonne méthode pour évaluer des trajectoires afin de déterminer les plus intéressantes.
		
		j'ai décidé de chercher à optimiser les observations indépendamment les unes des autres puis de calculer des trajectoire viable d'une observation à l'autre.
		
		\subsection{calcul des observations}
		ensuite j'ai implémenté le modèle dans Matlab ainsi qu'une méthode permettant de calculer le temps passé dans la zone d'observation en fonction d'une vitesse et position initiale situé dans la zone d'observation.\\
		
		J'ai pu obtenir de premiers résultat, résultats qui ont permis de mettre en évidence deux points de l'orbite lunaire qui semblait être assez intéressant pour effectuer des observations : ces deux points se situant aux moments où Soleil, la Terre et la Lune forment un angle d'environ 60 degrés.
		
		
		\begin{figure}[h]
			\includegraphics[width=10cm]{images/observations_main.png}
			\caption{orbite keplerienne qui permetent de faire des observations aux points d'interêt mentionné plus tôt}
		\end{figure}
		
		%peut être une démo mathématique ici ?
		
		J'ai également amélioré le modèle en prenant en compte de plus en plus de facteur comme l'excentricité et l'inclinaison de l'orbite de la Lune, la gravité de la Lune qui influence la dynamique du satellite ainsi que les variations du mouvement de la lune causé par la gravité solaire.
		
		Aux termes du projet, j'ai réalisé un code MATLAB d'une longueur approximativement 600 lignes qui effectue de nombreuses opérations décrite ci dessous: 
		
		\begin{itemize}
			\item simuler la dynamique du système à trois corps Soleil Lune Terre (l'influence de du système Terre-Lune sur le soleil est négligé);
			\item obtenir des approximation des emplacements des observations optimale dans le but de trouver de bonnes conditions initiales à l'aide d'un modèle simplifié;
			\item optimiser les observations optimale pour chaque minimum trouvé à l'étape précédente
			\item obtenir des images graphiques et des fichiers csv contenant les données.
		\end{itemize}
		
		Le code à besoin en entrée de multiples informations sur la dynamique du problème dont la plupart des facteurs sont fournis directement dans le code. La partie venant de l'extérieur du code sont les positions et vitesses initiales du soleil de la Lune et de La Terre ainsi que l'intervalle de temps sur lequel il faut rechercher des observations.
		\\ \\
		Ce code pourrait être facilement adapté en librairie
		\\ \\	
		Ce travail permis de confirmer l'intérêt des deux points de l'orbite lunaire qui semblait être les plus intéressants avec des temps d'observation situé entre 15 et 25 heures. En sachant qu'il y a deux observations de ce type pour chaque période synodique lunaire (environ 29.53 jours), cela permettrait de faire des observations pendant environs 2.8\% du temps (ou 1.4\% du temps en n'en faisant qu'une sur deux) de la mission ce qui est bien mieux que ce que l'on peut faire sur Terre (une observation de 10 minutes tout les 18 mois en moyenne soit environs 0.0015\% du temps).
		
		le fait que les temps d'observation soit bien plus long aux points spécifique où le satellite est lancé depuis la zone d'observation m'a poussé à faire un test dans lequel le satellite étant toujours lancé à la même vitesse que la zone d'observation sans se soucier de sa période, ce test à bien confirmer que les 2 observations trouvé précédemment sont bien les plus efficace mais a permis de mettre en évidence un $3^{ème}$ point de l'orbite lunaire aux alentour de $0^o$. Cependant, cette trajectoire n'est pas idéale car, en plus ne pas avoir une période similaire à celle de la lune, l'observation a lieu à proximité de la zone d'ombre de la Terre ce qui rendrait toutes observations impossible si la zone d'observation traversais celle de la Terre. De plus, l'excentricité de l'orbite associé à cette trajectoire est aux alentour de 1 ce qui implique que le satellite pourrait être facilement quitter le système Terre Lune ce qui serait embêtant.
		\begin{figure}[h]
			\includegraphics[width=1\textwidth]{images/observation_Obs.png}
			\caption{temps d'observation en fonction de l'anomalie vraie de la Lune}
		\end{figure}
		
		\subsection{contrôle optimal}
		
		l'étape suivante consistait à calculer la trajectoire du satellite pour joindre les différentes observations.
		pour cela j'ai utilisé une librairie en Julia permettant de résoudre des problème de contrôle optimal, l'idée était de calculer tout les transferts possibles entre différentes observations.
		
		j'ai implémenté deux versions du problème, l'une d'entre elle cherche à minimiser l'énergie cinétique lors d'un transfert, tandis que l'autre cherche à minimiser l'accélération maximale lors du transfert
		
		Après avoir implémenté une version simplifiée du problème des voiles solaires (poussée faible dans n'importe quelle direction), il se trouve que les accélérations demandée semblait trop élevée pour utiliser une voile solaire, l'option d'utiliser un satellite à poussée faible a donc été sélectionné.
		
		concernant les transferts, j'ai effectué une analyse des résultats qui m'ont permis de sélectionner 2 scénarios possible de mission dans lesquels un satellite effectue une observation sur 2 en utilisant une poussée faible.
		
		suite à ces premier calcul j'ai mis à jour le programme pour prendre en compte l'équation de la masse, j'ai utilisé comme moteur le RIT-10 evo.
		
		ce nouveau code à ensuite été rafiné pour effectué les opérations suivantes : 
		
		\begin{itemize}
			\item simuler la dynamique du système à trois corps Soleil Lune Terre (l'influence de du système Terre-Lune sur le soleil est négligé);
			\item définir et résoudre un modèle de contrôle optimal de transfert avec bord fixé en minimisant l'énergie dépensée.
			\item répéter l'opération pour plusieurs scénario 
		\end{itemize}
		
		\subsection{analyse de résultat}
		
		à plusieurs moments durant ce stage j'ai du commenter des résultats de calculs, j'ai du rechercher les raisons et les causes qui ont mené à certain résultat inattendus ainsi que de déduire des potentiels calculs supplémentaire ou les prochaines étapes du projet. Pour commencer j'ai utilisé les résultats du solver qui trouvait les plus long temps d'observation pour isoler des les points les plus intéressants pour faire des observations. Suite à ces analyse j'en ai conclu que ce qui semblait le plus intéressant était d'optimiser les observations indépendamment les unes des autres puis de chercher par la suite des trajectoire de transfert qui permettrait de passer d'une orbite à l'autre.
		
		J'ai également du faire de choix en fonctions de ces résultat. J'ai par exemple remarqué qu'en prenant une période de révolution plus courte pour le satellite ( de l'ordre de $0.92$ période lunaire au lieu de $1$) les temps d'observation obtenue était plus court. Je pense que la cause de ces résultat vient du fait que la contrainte de vitesse lié à la période fixée du satellite est de trop, cependant donner au satellite la liberté d'avoir n'importe quelle vitesse en partant des optimums trouvé avec le modèle simplifié donne de très mauvais résultat. J'ai donc décidé d'optimiser d'abord avec la norme de vitesse fixé par la période du satellite (fixé à $0.92$) puis de résoudre à nouveau avec un modèle modifié laissant la vitesse libre en prenant comme condition initiale le résultat de l'algorithme précédent. Il est probable qu'il faillent remplacer le modèle simplifié par un autre plus performant. Un qui prendrait en compte le temps d'observation en fixant la vitesse de l'objet à la vitesse de la zone d'observation comme dans la figure 2 par exemple. Mais la forme assez complexe des optimums (deux pics très rapproché) et le fait que les observations des résultats ont rarement montré des trajectoire à vitesse nulle au centre de la zone (les vitesses sont souvent nulles près des bord de la zone) ont fait que je n'ai pas eu le temps d'implémenter un autre modèle.
		
		Ensuite j'ai du analysé les résultats de mon second code fait en Julia pour essayé de comprendre pourquoi les accélérations maximale des transferts ne diminuait pas même en faisant des transferts sur plus longtemps.
		Il s'avère que, si l'on considère un modèle simplifié d'une Lune et d'une Terre avec une orbite circulaire sur le même plan, les orbites d'observations sont sensiblement les mêmes avec pour unique différence une rotation autour du vecteur normal au plan écliptique. Cette rotation est proportionnelle aux temps entre deux observation. Cela signifie que plus deux observations sont éloigné plus il faudra faire tourner l'orbite d'observation sur un grand angle ce qui explique que la les accélérations maximales ne diminuaient pas. Effectuer un transfert sur une observation deux fois plus lointaine dans le temps nécessite de faire tourner l'orbite d'observation deux fois plus loin également. C'est cette conclusion suggérant qu'il y a un seuil d'accélérations minimal que doit atteindre le satellite pour effectuer un nombre important d'observations par an qui à mené à l'abandon des voiles solaires au profit d'un moteur à poussée faible.
		
		\newpage
		\section{Conclusion}
		\subsection{compétence acquise}
		Pour conclure ce stage m'a permis d'obtenir de nombreuses compétences.
		 J'ai eu l'occasion de maîtriser les bases du contrôle optimal. Ce stage m'a également permis de mettre en application ces connaissances en utilisant une librairie permettant de résoudre des problèmes de ce type.
		
		J'ai également renforcé mes compétences en programmation en apprenant un nouveau langage de programmation : le Julia qui est un langage utilisé en recherche pour le calculs scientifique et la résolution numérique de problème.J'ai également grandement amélioré mes connaissances dans le langage de programmation Matlab qui est lui aussi un langage utilisé pour la résolution numérique de problèmes.
		
		le fait de programmer en deux langages différent m'a forcé faire transiter les résultats d'un programme à l'autre en utilisant des fichiers csv ce qui m'a permis de faire des applications plus propres et à renforcé mes capacité à créer des codes pouvant être inclus et utilisés dans d'autres applications que celles pour lesquelles ils avaient été initialement programmés.
		
		J'ai eu également l'occasion d'écrire un compte-rendu de projet (qui n'est pas ce rapport de stage) intégralement en Anglais ce qui, en plus de renforcé mes compétences en Anglais, m'a permis de devenir plus doué dans la rédaction de rapports scientifiques en vue de publication.
		
		
		
\end{document}